% based on a file Copyright 2007 by Till Tantau
% This file may be distributed and/or modified
% 1. under the LaTeX Project Public License and/or
% 2. under the GNU Public License.
\documentclass[handout]{beamer}
\setbeamertemplate{footline}[frame number]
\usetheme[progressbar=frametitle]{metropolis}
\makeatletter
\setlength{\metropolis@titleseparator@linewidth}{3pt}
\setlength{\metropolis@progressonsectionpage@linewidth}{3pt}
\setlength{\metropolis@progressinheadfoot@linewidth}{3pt}

\usecolortheme{rose}
\xdefinecolor{pur}{rgb}{.25,.2,.80}
\usepackage{amssymb,amsmath,amsthm}
\newcommand{\purco}[1]{{\color{pur}{#1}}}
\newcommand{\purline}{${\color{pur}\underline{\hspace{4.3in}}}$}
\newtheorem{defn}{Definition}
\newtheorem*{thm}{Theorem}
\newtheorem*{coro}{Corollary}
\newtheorem*{rmk}{Remark}
\newtheorem*{question}{Question}
\newtheorem*{conjecture}{Conjecture}

\setbeamertemplate{navigation symbols}{}
\usepackage{mathrsfs}

% Standard packages

\usepackage[english]{babel}
\usepackage[latin1]{inputenc}
%\usepackage{times}
%\usepackage[T1]{fontenc}
\usepackage{hyperref}
\usepackage{epsfig}
\usepackage{graphicx}
\graphicspath{ {./GraphsFamilies/} }
%\usepackage{tikzlings}
\usepackage{comment}
\usepackage{appendixnumberbeamer}
%\newcommand{\db}{\overline{d}}
\newcommand{\burl}[1]{\textcolor{blue}{\url{#1}}}
\newcommand{\emaillink}[1]{\textcolor{blue}{\href{mailto:#1}{#1}}}

\newcommand{\blue}[1]{ {\color[rgb]{0,0,.85} #1 } }
\newcommand{\red}[1]{ {\color[rgb]{.55,0,.15} #1 } }
\newcommand{\marroon}[1]{ {\color[rgb]{.55,.0,.5} #1 } }
\newcommand{\green}[1]{ {\color[rgb]{0,.85,0} #1 } }

%%make section title slides show up (added by Roger)

\AtBeginSection[]{
  \begin{frame}
  \vfill
  \centering
  \begin{beamercolorbox}[sep=8pt,center,shadow=true,rounded=true]{title}
    \usebeamerfont{title}\insertsectionhead\par%
  \end{beamercolorbox}
  \vfill
  \end{frame}
}

\newcommand{\mcs}{\mathcal{S}}
\newcommand{\mcd}{\mathcal{D}}
\newcommand{\mcdc}{\mathcal{D}^{\rm c}}
\newcommand{\mcsc}{\mathcal{S}^{\rm c}}
\newcommand{\mel}{\mathcal{M}}
\newcommand{\ncr}[2]{{#1 \choose #2}}
\newcommand{\un}{\text{U}}
\newcommand{\sy}{\text{USp}}
\newcommand{\usp}{\text{USp}}
\newcommand{\soe}{\text{SO(even)}}
\newcommand{\soo}{\text{SO(odd)}}
\newcommand{\so}{\text{O}}
\newcommand{\sym}{\text{sym}}

\newcommand{\fon}{\frac{1}{N}}
\newcommand{\fonk}{ \frac{1}{N^{k+1}}}
\newcommand{\glik}{\gl_i^k(A)}
\newcommand{\mtk}{M_{2k}(N)}
\newcommand{\bs}[2]{b_{|i_{#1}-i_{#2}|}}
\newcommand{\Inv}[1]{\frac{1}{#1}}


\renewcommand{\i}{{\mathrm{i}}} % sqrt -1

\newcommand{\twocase}[5]{#1 \begin{cases} #2 & \text{#3}\\ #4
&\text{#5} \end{cases}   }

\newcommand{\threecase}[7]{#1 \begin{cases} #2 & \text{{\rm #3}}\\ #4
&\text{{\rm #5}}\\ #6 & \text{{\rm #7}} \end{cases}   }


\newcommand{\lp}{\left(}
\newcommand{\rp}{\right)}


\renewcommand{\i}{{\mathrm{i}}} % sqrt -1

\newcommand\be{\begin{equation}}
\newcommand\ee{\end{equation}}
\newcommand\bea{\begin{eqnarray}}
\newcommand\eea{\end{eqnarray}}
\newcommand\bi{\begin{itemize}}
\newcommand\ei{\end{itemize}}
\newcommand\ben{\begin{enumerate}}
\newcommand\een{\end{enumerate}}


\newcommand{\notdiv}{\nmid}

\newcommand{\pr}{\mathbb{P}}
\newcommand{\R}{\mathbb{R}}
\newcommand{\C}{\mathbb{C}}
\newcommand{\Z}{\mathbb{Z}}
\newcommand{\Q}{\mathbb{Q}}
\newcommand{\N}{\mathbb{N}}

\DeclareMathOperator{\nsmod}{mod}
\newcommand{\nospacemod}[1]{\nsmod#1}

\newenvironment{variableblock}[3]{%
  \setbeamercolor{block body}{#2}
  \setbeamercolor{block title}{#3}
  \begin{block}{#1}}{\end{block}}

\definecolor{lightcyan}{RGB}{240,255,255}

% Author, Title, etc.

\Large

\title[PLRS]
{
  Introduction to Completeness of Generalized Fibonacci Sequences
}

\author[Steven J Miller] %%
{
    \textcolor{black}{
        El\.zbieta Bo\l dyriew (\href{mailto:eboldyriew@colgate.edu}{eboldyriew@colgate.edu}) \\
        John Haviland (\href{mailto:havijw@umich.edu}{havijw@umich.edu}) \\
        Ph\'uc L\^am (\href{mailto:plam6@u.rochester.edu}{plam6@u.rochester.edu}) \\
        John Lentfer (\href{mailto:jlentfer@hmc.edu}{jlentfer@hmc.edu}) \\
        Fernando Trejos Su\'arez (\href{mailto:fernando.trejos@yale.edu}{fernando.trejos@yale.edu}) \\
        Advised by: Steven J.\ Miller (\href{mailto:sjm1@williams.edu}{sjm1@williams.edu}) \\
        \textcolor{cyan}{Research conducted as part of the 2020 SMALL Research Experience for Undergraduates at Williams College}
    }
}

\date[2020]
{\small{\textcolor{orange}{Combinatorial and Additive Number Theory (CANT 2021)\\
05/26/21}}
	%\alert{College Name, Date, Year}
}


% The main document

\begin{document}

\begin{frame}
  \titlepage
\end{frame}

\large

%%%%%%%%%%%%%%%%%%%%%%% INTRO %%%%%%%%%%%%%%%%%%%%%%%%%%%%%%%%%%%%%%%%%%%

\section{Introduction}
\begin{frame}{Motivation}
\begin{itemize}
    \item \textbf{Positive linear recurrence sequences} (PLRS) generalize the Fibonacci numbers in Zeckendorf's theorem.
        \pause\bigskip
    \item \textbf{Complete} sequences can be used to to express integers using sums of terms.
        \pause\bigskip
    \setbeamercolor{block}{bg=yellow, fg=white}
    \begin{variableblock}{Research Question}{bg=lightcyan,fg=black}{bg=cyan,fg=white}
    How can we determine whether a PLRS is complete based on the coefficients in its defining recurrence relation?
    \end{variableblock}
\end{itemize}
\end{frame}

\begin{frame}{
% \begin{tikzpicture}[scale=0.5]\tikzling[signpost=\scalebox{0.6}{1},\randomhead] \end{tikzpicture}
Positive Linear Recurrence Sequences}
    \begin{definition}
        A sequence $\{H_i\}_{i \ge 1}$ of positive integers is a {\bf Positive Linear Recurrence Sequence (PLRS)} if: 
        \begin{itemize}
            \pause
            \item (Recurrence relation) There are non-negative integers $L, c_1, \dots, c_L$ such that \vspace{-0.1in}
                \[ 
                    H_{n+1} = c_1H_n + \dots + c_LH_{n+1-L}
                \]
            with $L, c_1, c_L$ positive.
            \pause
            \item (Initial conditions) $H_1 = 1$, and for $1 \le n \le L$, \vspace{-0.1in}
                \[
                    H_{n+1} = c_1H_n + \cdots + c_nH_1 + 1
                \]
        \end{itemize}
    \end{definition}
\end{frame}

\begin{frame}{ 
%\begin{tikzpicture}[scale=0.5]\tikzling[signpost=\scalebox{0.6}{1},\randomhead]\end{tikzpicture}
    Positive Linear Recurrence Sequences}
    \begin{itemize}
        \item Write $[c_1, \ldots ,c_L]$ for $H_{n+1} = c_1 H_n + \cdots + c_L H_{n-L+1}$.\\ \
        \pause
        %\item The definition requires that $c_1, \ldots ,c_L$ must be all non-negative, with the first and last entries positive. 
        %\item The initial conditions are specified to give unique decompositions (by a generalized Zeckendorf Theorem)
        \item Fibonacci numbers: $[1,1]$. Initial conditions $F_1=1,\; F_2=2$.\\ \
        \pause
        \item (Lucas and Pell numbers are not PLRS, due to initial conditions).
    \end{itemize}
\end{frame}


%\section{Completeness of PLRS}
\begin{frame} {
%\begin{tikzpicture}[scale=0.5]\tikzling[signpost=\scalebox{0.6}{2},\randomhead]\end{tikzpicture} 
    Introduction to Completeness}
    \begin{definition}
        A sequence $\{H_i\}_{i \ge 1}$ is \textbf{complete} if every positive integer is a sum of its terms, using each term at most once.
    \end{definition}
    \medskip
    \begin{itemize}
    \pause
     \item The sequence $[1, 3]$ is \emph{not} complete. Its terms are $\{1,2,5,11,\dots\}$; you cannot get $4$ or $9$.\medskip
    \pause
    \item The Fibonacci sequence is complete (follows from Zeckendorf's Theorem).
    \end{itemize}
    
    
\end{frame}

\begin{frame}{The Doubling Sequence $H_{n+1} = 2H_n$}
    The PLRS $[2]$ has terms $H_n = 2^{n - 1}$, i.e., $\{1,2,4,8,\ldots\}$, and is complete (every integer has a binary representation).
    \pause
    \bigskip
    \begin{theorem}[Brown]
    The complete sequence with maximal terms is $H_n= 2^{n-1}$.
    \end{theorem}
        %\begin{theorem}
        %The sequence $a_n = 2^{n - 1}$ is the only complete sequence for which the representation of every positive integer is unique.
        %\end{theorem}
    \pause
    \bigskip
    Any PLRS of the form $[1, \ldots, 1, 2]$ has the same terms as $[2]$, i.e., $H_n = 2^{n - 1}$.
    %For any $m$, the sequence defined by $[\underbrace{1,\ldots,1}_{m},2]$ is identical to the sequence defined by $[2]$.
    %this sequence, namely $\forall n, H_n=2^{n-1}$.
\end{frame}
%\begin{frame}{Legal Decompositions vs. Completeness}
%    \begin{itemize}
        %\item By Zeckendorf's Theorem, any positive integer can be written uniquely as a sum of non-adjacent Fibonacci numbers. Thus, they are also complete.
%        \item Previous work on PLRS relates to \emph{legal decompositions}, which are another way to write integers as sums of sequence terms.
%        \item Given any PLRS, there is a legal decomposition of every positive integer. Does this mean that all PLRS are complete?
        %All PLRS have a type of unique decomposition for all positive integers, called legal decompositions. Does this mean all PLRS are complete?
%        \pause
%        \item No. For legal decompositions, sequence terms can be used more than once. This is not allowed for completeness decompositions.
        %elements of the sequence can be used multiple times.
%    \end{itemize}
    
%    \pause
    
%    \begin{example}
%        The PLRS $[1,3]$ has terms $1, 2, 5, 11, \ldots$. The unique \emph{legal} decomposition for $9$ is $5 + 2(2)$, where the term $2$ is used twice. However, no \emph{complete} decomposition for $9$ exists.
%    \end{example}
%\end{frame}

\begin{frame}{
%\begin{tikzpicture}[scale=0.5]\tikzling[signpost=\scalebox{0.6}{3},\randomhead]\end{tikzpicture}
    Brown's Criterion}
    \begin{theorem}[Brown]
    A nondecreasing sequence $\{H_i\}_{i \ge 1}$ is complete if and only if $H_1 = 1$ and for every $n \ge 1$,\vspace{-0.1in}
        \[
            H_{n + 1} \ \leq \ 1 + \sum_{i = 1}^{n} H_i.
        \]
    \end{theorem}
    \pause
    \bigskip
    \begin{definition}
    The \textbf{$\boldsymbol{n}$-th Brown's Gap} of a sequence $\{H_i\}_{i \ge 1}$ is $$B_{H, n} \ := \ 1 + \left(\sum_{i=1}^{n-1}H_i\right) - H_n .$$
    \end{definition}
\end{frame}

\begin{comment}
\begin{frame}{Complete Sequences in Practice}
    \begin{definition}
    A \textbf{numeration system} is a set of integer basis elements where every integer can be represented uniquely over the set using bounded integer digits.
    \end{definition}
    \begin{itemize}
        \item Numeration systems facilitate problems in combinatorial group theory, ranking of permutations and combinations, strategies of games, and many others. 
        \item Previous work in numeration systems defined by recurrent sequences was purely combinatorial and heavily based on the words corresponding to greedy expressions for natural numbers. 
    \end{itemize}
\end{frame}
\end{comment}

\section{Modifying Sequences}


\begin{frame}{
%\begin{tikzpicture}[scale=0.5]\tikzling[signpost=\scalebox{0.6}{7},\randomhead]\end{tikzpicture}
    Example for $L=6$}
    \begin{example}
    $[1,0,0,0,0,N]$ is complete if and only if $N \leq 11$.
    \end{example}
    \bigskip
    \pause

    \definecolor{lightcyan}{RGB}{240,255,255}
    \begin{variableblock}{Question}{bg=lightcyan,fg=black}{bg=cyan,fg=white}
    Is there another choice of coefficients $[c_1,\ldots,c_5, N]$, that generates a complete PLRS, with some $N > 11$?
    \end{variableblock}
        %\item Is this the highest $N$ we can get for any PLRS of length $L=6$?
        %\item By the second part of the theorem, $[1,1,0,0,0,N]$ is complete for $N \leq 11$ by switching the first zero to a one!
\end{frame}

%blank slide?

\begin{frame}{
%\begin{tikzpicture}[scale=0.5]\tikzling[signpost=\scalebox{0.6}{7},\randomhead]\end{tikzpicture}
    Example for $L=6$}
%Here are the maximal last terms for preserving completeness for several other sequences of length $L=6$:
\pause
\begin{itemize}
    \item $[1,0,0,0,0,N]$ is complete for $N \leq 11$.
    \item $[1,1,0,0,0,N]$ is complete for $N \leq 11$.
    \item $[1,0,1,0,0,N]$ is complete for $N \leq 12$.
    \item $[1,0,0,1,0,N]$ is complete for $N \leq 11$.
    \item $[1,0,0,0,1,N]$ is complete for $N \leq 10$.
\end{itemize}
\pause
Why is $[1,0,1,0,0,12]$ complete, but $[1,0,0,0,0,12]$ is not complete? 
    
\end{frame}

\begin{frame}{
%\begin{tikzpicture}[scale=0.5]\tikzling[signpost=\scalebox{0.6}{7},\randomhead]\end{tikzpicture}
    Example for $L=6$}
Why is $[1,0,1,0,0,12]$ complete, but $[1,0,0,0,0,12]$ is not complete? \\
\begin{itemize}
\pause
\item $[1,0,0,0,0,12]$ has terms $\{1,2,3,4,5,6,18,\textcolor{red}{42},\dots \}$\\
         and so computing $1 + \sum_{i=1}^{n}H_i$ we see $\{2,4,7,11,16,22,\textcolor{red}{40},\dots \}$\\    
\pause
\item $[1,0,1,0,0,12]$ has terms $\{1,2,3,5,8,12,29,\textcolor{blue}{61},\dots \}$\\
         and so computing $1 + \sum_{i=1}^{n}H_i$ we see $\{2,4,7,12,20,32,\textcolor{blue}{61},\dots \}$\\
\pause
\item $[1,1,1,0,0,12]$ has terms $\{1, 2, 4, 8, 15, 28,\textcolor{red}{63},\dots \}$\\
         and so computing $1 + \sum_{i=1}^{n}H_i$ we see $\{2,4,8,16,31,\textcolor{red}{59},\dots \}$\\
% Talk about jump in sequence when last coefficient comes into play
%\item In order to maximize $N$, a complete sequence cannot grow too fast; however, by increasing intermediate terms, it is possible to give the partial sums a boost which will allow for a larger final coefficient.
\end{itemize}
\end{frame}

\begin{frame}{Modifying Coefficients of a PLRS}
What modifications to the coefficients preserve completeness or incompleteness?
\pause
	\begin{theorem}[SMALL 2020]
	If $[c_1,\ldots , c_{L}]$ is any incomplete sequence, then the sequence $[c_1,\ldots ,c_{L-2}, c_{L-1}+c_{L}]$ is also incomplete.
\end{theorem}
%\begin{theorem}\label{lem:incompAddCoeff}
 %   Consider sequences $\left( G_{n} \right) = [c_1,\dots, c_{L}]$ and $\left( H_{n} \right)= [c_1,,\dots, c_{L},c_{L+1}]$, where $c_{L+1}$ is any positive integer. If $\left( G_{n} \right)$ is incomplete, then $\left( H_{n} \right)$ is incomplete as well.
%\end{theorem}
\pause
\bigskip
\begin{theorem}[SMALL 2020]
If a sequence $[c_1, \ldots, c_{L-1}, c_L]$ is complete, then so is $[c_1, \ldots, c_{L-1}, d_L]$ for any $1 \le d_L \leq c_L$.\\
    \emph{Remark}. Not true for $c_i$ in an arbitrary position.

\end{theorem}
We discuss bounds for the last coefficient.% when we keep the first $L-1$ coefficients fixed.
\end{frame}

\section{Families of Sequences}

\begin{frame}{
%\begin{tikzpicture}[scale=0.5]\tikzling[signpost=\scalebox{0.6}{4},\randomhead]\end{tikzpicture}
    Analyzing Families of Sequences}
    \begin{theorem}[SMALL 2020]
        \begin{itemize}
        \item     
        $[1, \underbrace{0, \ldots, 0}_k, N]$, is complete if and only if \vspace{-0.2in}
            \[
            N \ \leq \ \left\lfloor \frac{(k + 2)(k + 3)}{4} + \frac{1}{2} \right\rfloor.
            \]
        \item $[1, 1, \underbrace{0, \ldots, 0,}_k N]$, is complete if and only if
        \vspace{-0.2in}
            \[
            N\leq \left\lfloor\frac{F_{k+6}-(k+5)}{4}\right\rfloor,
            \]
            where $F_k$ is the $k$th Fibonacci number.
        \end{itemize}
    \end{theorem}
\end{frame} 

\begin{frame}{
%\begin{tikzpicture}[scale=0.5]\tikzling[signpost=\scalebox{0.6}{5},\randomhead]\end{tikzpicture}
    Proof Sketch}
    \begin{theorem}
        $[1, 0, \ldots, 0, N]$, with $k$ zeros, is complete if and only if
            $N \leq \left\lfloor \frac{(k + 2)(k + 3)}{4} + \frac{1}{2} \right\rfloor$.
    \end{theorem}
    \pause
    \emph{Partial Proof.}
        We sketch that if $N_{\max} = \left\lfloor \frac{(k + 2)(k + 3)}{4} + \frac{1}{2} \right\rfloor$, then the sequence is complete.\\ %For $N < N_{\max}$, we can apply the previous theorem to preserve completeness.
        \pause
        With the recurrence relation and Brown's criterion,
        \begin{align*}
            H_{n+1} &=H_n + N_{\max}H_{n-k-1}\\
        &\leq H_n +(N_{\max}-1)H_{n-k-1} + H_{n-k-2} + \dots + H_1 +1
        \intertext{By induction, $(N_{\max}-1)H_{n-k-1}\leq H_{n-1} +\dots+H_{n-k-1}$, so}
         &\leq H_n + \dots + H_1 +1.
        \end{align*}
\end{frame}

\begin{frame}{}
    
    \begin{figure}
    \centering
    \includegraphics[scale=0.6]{middle.eps}
    \caption{$[\protect1,\underbrace{0, \dots, 0}_{k},1, \protect\underbrace{0, \dots, 0}_{g}, N]$ with location of middle one varying, where each color represents a fixed length $L$.}
    \label{fig:MiddleOne}
\end{figure}
\end{frame}

\begin{frame}{Theorem on Switching Ones}
\begin{theorem}[SMALL 2020]
Let $L \geq 6$ fixed and $\{ H_n\}= [1,\underbrace{0,\dots,0}_{L-g-3} ,1, \underbrace{0,\ldots,0}_{g},M]$,  $0 < g \le L-3$. If $M$ is maximal such that $\{H_n\}$ is complete, and $N$ is maximal such that 
$[1,0,\dots,0,N]$ is complete, $M\geq N$.

In particular,
\begin{itemize}
    \item $[1, 0, \dots, 0, 0, 1, M]$ is complete if and only if $M \le N-1$
    \item $[1, 0, \dots, 0, 1, 0, M]$ is complete if and only if $M \le N$.
\end{itemize}
\end{theorem}
\end{frame}

\begin{frame}{}
    \begin{figure}
    \centering
    \includegraphics[scale=0.6]{nbonacci.eps}
    \caption{$[\protect\underbrace{1, \dots, 1}_{g}, \protect\underbrace{0, \dots, 0}_{k}, N]$ with $k$ and $g$ varying, where each color represents a fixed $k$.}
    \label{fig:FamiliesOfOne}
\end{figure}
\end{frame}

\begin{frame}{
%\begin{tikzpicture}[scale=0.5]\tikzling[signpost=\scalebox{0.6}{7},\randomhead]\end{tikzpicture}
    Sequences of Initial Ones}
\begin{theorem}[SMALL 2020]
If a sequence $[\underbrace{1,\ldots,1}_g,\underbrace{0,\ldots,0}_k,N]$ is complete with $g\geq 3,$ then
%\setlength{\abovedisplayskip}{0pt}
%\setlength{\belowdisplayskip}{0pt}
% The following line would not compile for me
%\[\resizebox{1\textwidth}{!}{N \leq \frac{1}{2} \left( 1 + \sum\limits_{i = 1}^{k + 1}F_i^{(m)} + \sum\limits_{i = 1}^{k + 1 - m}F^{(m)}_{i} + \cdots + \sum\limits_{i = 1}^{(k + 1)\nospacemod{m}}F^{(m)}_{i} \right)}\]
\[{N \leq \frac{1}{2} \left( 1 + \sum\limits_{i = 1}^{k + 1}F_i^{(g)} + \sum\limits_{i = 1}^{k + 1 - g}F^{(g)}_{i} + \cdots + \sum\limits_{i = 1}^{(k + 1)\nospacemod{g}}F^{(g)}_{i} \right)}\]
where $F_{i}^{(g)}$ is the $g$-bonacci sequence, $[\underbrace{1,\ldots,1}_g]$.
\end{theorem}
\end{frame}

\begin{frame}{Sequences of Initial Ones}
\begin{alertblock}{Conjecture (SMALL 2020)}
If a sequence $[\underbrace{1, \ldots,1}_g,\underbrace{0,\ldots,0}_k, N]$  is complete, then so is  $[\underbrace{1, \ldots,1}_{g+j},\underbrace{0,\ldots,0}_k, N]$ for any positive integer $j.$
\end{alertblock}

\pause
\begin{theorem}[SMALL 2020]
Consider $[\underbrace{1, \ldots,1}_g,\underbrace{0,\ldots,0}_k, N]$.
\begin{itemize}
    \item For $g \ge k + \lceil \log_2k \rceil$, the bound on $N$ is $N \le 2^{k+1}-1$
    \item For $k \le g < k + \lceil \log_2k \rceil$, the bound on $N$ is $N \le 2^{k+1} - \left\lceil \dfrac{k}{2^{g-k}} \right\rceil$
\end{itemize}
\end{theorem}
\end{frame}

\begin{comment}
\begin{frame}{Idea of Proof}
\begin{theorem}[SMALL 2020]
\begin{itemize}
    \item For $g \ge k + \lceil \log_2k \rceil$, the bound on $N$ is $N \le 2^{k+1}-1$
    \item For $k \le g < k + \lceil \log_2k \rceil$, the bound on $N$ is $N \le 2^{k+1} - \left\lceil \dfrac{k}{2^{g-k}} \right\rceil$
\end{itemize}
\end{theorem}
\pause
\begin{theorem}[SMALL 2020]
    The PLRS $\{H_i\}_{i \ge 1}$ generated by $[c_1, \dots, c_L]$ is complete if $$\begin{cases} B_{H, n} \ge 0, \; 1 \le n < L \\ B_{H, n} > 0, \; L \le n \le 2L-1 \end{cases}$$
\end{theorem}
\end{frame}
\end{comment}

\section{The $2L - 1$ conjecture}
\begin{frame}{The $2L - 1$ conjecture}
    Can we bound where a sequence must fail Brown's Criterion?
    \pause \alert{We think so!}
    \bigskip
    \begin{alertblock}{Conjecture (SMALL 2020)}
        If a PLRS $H_{n + 1} = c_1 H_n + \cdots + c_L H_{n + 1 - L}$ incomplete, then it fails Brown's criterion before the $2L$-th term.
    \end{alertblock}
\pause
The closest we've gotten:
%In other words, if $B_{H, n} \ge 0$ for all $1 \le n \le 2L-1$, then $\{H_i\}_{i \ge 1}$ is complete.
\begin{theorem}[SMALL 2020]
    The PLRS $\{H_i\}_{i \ge 1}$ generated by $[c_1, \dots, c_L]$ is complete if $$\begin{cases} B_{H, n} \ge 0, \; 1 \le n < L \\ B_{H, n} > 0, \; L \le n \le 2L-1 \end{cases}$$
\end{theorem}
\end{frame}

\begin{frame}{}
\begin{figure}
    \centering
    \includegraphics[scale=0.5]{24slide.eps}
    \caption{$[1,\protect\underbrace{0, \dots, 0}_{k}, \protect\underbrace{1, \dots, 1}_{m}, N]$ with number of ones ($m$) varying, depending on $L$.}
    \label{fig:LastOnes}
\end{figure}
\end{frame}

\begin{frame}{
%\begin{tikzpicture}[scale=0.5]\tikzling[signpost=\scalebox{0.6}{8},\randomhead]\end{tikzpicture}
    Conditional result on Adding Ones}
    If the $2L-1$ conjecture holds, we have the following:
    \begin{alertblock}{Theorem (SMALL 2020)}
For a fixed length $L$, the sequence $[1,\underbrace{0,\ldots, 0}_{k},\underbrace{1,\ldots,1}_{m},N]$ with $m$ ones has a lower bound on $N$ than the sequence $[1,\underbrace{0,\ldots, 0}_{k-1},\underbrace{1,\ldots,1}_{m+1},N]$.

In particular, if $m<\frac{L}{2}$, the bound is precisely
\begin{center}\resizebox{0.9\textwidth}{!}{$\displaystyle N \leq \left\lfloor \frac{\left( L-m \right) \left( L+m+1 \right) }{4}+\frac{1}{48}m(m+1)(m+2)(m+3)+\frac{1-2m}{2} \right\rfloor.$}\end{center}
\end{alertblock}
\end{frame}

\section{Binet's Formula and Generalizations}

\begin{frame}{Characteristic Polynomials}
\begin{definition}
	For a PLRS $\{ H_{n} \}$ defined by $[c_1,\ldots , c_{L}]$,  define the characteristic polynomial
	\[
	p(x)=x^{L}-\sum_{i=1}^{L}c_{i}x^{L-i}
	.\] 
\end{definition}
\pause
\begin{itemize}
\item
 By Descartes's Rule of Signs, $p(x)$ there is one positive real root, the \textbf{principal root}. 
 \pause
\item
The principal root is always the largest: for any root $z\in \C,$ $\left| z \right|<r$. 
\end{itemize}

\end{frame}

%\begin{frame}{Binet's Formula}
%\begin{theorem}[Binet]
%The terms $F_1,F_2,\ldots $ of the Fibonacci sequence are \[
%	F_{n}=\frac{1}{\sqrt{5}}\left( \varphi^{n}-\left( 1-\varphi  \right) ^{n} \right) 
%,\] 
%where $\varphi=\frac{1+\sqrt{5}}{2}$ denotes the Golden Ratio. 	
%\end{theorem}
%\pause
%\begin{itemize}
%\item
%	 $\varphi ,\; 1-\varphi $ are the roots of $p(x)=x^2-x-1$. 
%\end{itemize}
%\pause
%Can we get a similar result for a generic PLRS?
%\end{frame}

\begin{frame}{Generalized Binet's Formula}

\begin{theorem}[Generalized Binet's Formula]
If $r_1,\ldots , r_{k}$ are the roots of the polynomial of a linear recurrence $\{ H_{n} \}$ with multiplicities  $m_1,\ldots , m_{k}$, there are polynomials $q_1,\ldots , q_{k}$ with $\deg (q_i) \leq m_{i}-1$ such that \[
	H_{n}=q_1(n)r_1^{n}+\ldots +q_{k}(n)r_{k}^{n}
.\] 

\end{theorem}
\pause
	\begin{itemize}
\item
	If $\{ H_{n} \}$ is a PLRS, let $r_1$ be the principal root; since $m_1=1$ and for all $ i, r_1>|r_{i}|$, then $H_{n}=\Theta(r_1^{n})$.
	\item Complete sequences should grow ``slowly''. Can we relate the size of $r_1$ to completeness?
\end{itemize}
\end{frame}

%\begin{frame}{Slow- and Fast-Growing Sequences}
%\begin{itemize}
%\item
%	From Generalized Binet's Formula, we know $H_{n}=\mathcal{O}\left( r_1^{n} \right) $, so the asymptotic growth of $\{ H_{n} \}$ is determined by $r_1$.
%	\pause
%	\bigskip
%	\bigskip
%\item
%Complete sequences should grow ``slowly''. Can we relate the size of $r_1$ to completeness?
%\end{itemize}	
%\end{frame}
\section{Bounding the Principal Root}
\begin{frame}{First Bounds on $r_1$}
Recall $p(x)=x^{L}-\sum_{i=1}^{L}c_{i}x^{L-i}$.

	\pause
As  $c_{L} \geq 1$, we know $r_1>1$. ($c_{L}=\prod_{}^{}r_{i}^{m_{i}}$, and $r_1$ is the biggest root by magnitude). 

\pause
\begin{lemma}[SMALL 2020]
If $H_n$ is a complete PLRS and $r_1$ is its principal root, then $r_1\leq 2$.
\end{lemma}
\pause
\begin{proof}
	Otherwise, as $H_{n}=\Theta \left( r_1^{n} \right) $, for large $n$ our terms would exceed the maximal sequence $\{ 2^{n-1} \}$.
\end{proof}
	\pause
 Note: there are incomplete sequences with principal roots $r \leq 2$.


\end{frame}
\begin{frame}{Is 2 a Useful Bound?}
		%Is 2 the best upper bound for roots of complete sequences?
	%		\pause
	\begin{itemize}
	\item
		We can find complete sequences with roots of sizes arbitrarily close to 2. %and even with roots of size exactly 2. 
		(Sequences of the form $[\underbrace{1,\ldots,1}_{L}]$.)
	\pause
\item
Checking $r_1\leq 2$ is a fast method to eliminate candidates for completeness. %\pause How to do this effectively?
\pause
\item
	$p(x)=x^{L}-\sum_{i=1}^{L}c_{i}x^{L-i}$ has one positive real root, and $p(x)>0$ for large $x$, so $r_1\leq 2$ if and only if $p(2)\geq 0$. \pause This is much faster than checking Brown's Criterion!
	\end{itemize}
\end{frame}
\begin{frame}{Lower Bound}
	\begin{lemma}[SMALL 2020]
	For any $L $, there exists a second bound $B_{L}$ such that if a sequence $[c_1,\ldots , c_{L}]$ is incomplete, then  $r_1\geq B_{L}$.
\end{lemma}
\pause
\begin{proof}
	\begin{itemize}
	\item
		There are finitely many sequences $[ c_1,\ldots , c_{L} ]$ with $p(2)=2^{L}-\sum_{i=1}^{L}c_{i}2^{L-i}\geq 0$. % \pause  For example, if any $c_{i}> 2^{i}$, we have $p(2)< 0$.
		\pause
\item
	 Hence finitely many incomplete sequences with $r_1\leq 2$, so just find the minimum root -  $B_{L}$.\end{itemize} 
\end{proof}
\pause
We now aim to determine the precise values of $B_{L}$.
\end{frame}
%\begin{frame}{A Few Combinatorial Results}
%	\begin{theorem}[SMALL 2020]
%	If $[c_1,\ldots , c_{L}]$ is any incomplete sequence, then the sequence $[c_1,\ldots , c_{L-1}+c_{L}]$ is also incomplete.
%\end{theorem}
%\pause
%\bigskip
%	\begin{theorem}[SMALL 2020]
%If a sequence $[c_1, \ldots, c_{L-1}, c_L]$ is complete, then so is $[c_1, \ldots, c_{L-1}, d_L]$ for any $1 \le d_L \leq c_L$.\\
    %\emph{Remark}. This is not true for $c_i$ in an arbitrary position.
%\end{theorem}
%\pause
%Both can be proven by working directly with Brown's Criterion.
%\end{frame}
\begin{frame}{The Minimal Incomplete Sequence}
	    \begin{theorem}[SMALL 2020]
		    $[1, \underbrace{0, \ldots, 0}_{L-2}, N]$, is complete if and only if \vspace{-0.2in}
            \[
		    N \leq \left\lceil \frac{L(L+1)}{4}  \right\rceil.
            \]
	    \end{theorem}

	    \pause
\begin{conjecture}[SMALL 2020]
For any $L$, the incomplete sequence of length $L$ with smallest principal root is $[1, 0, \ldots, 0, \left\lceil \frac{L(L+1)}{4}\right\rceil+1]$.
\end{conjecture}
\pause
\begin{itemize}
\item

	Let $\lambda _{L}$ the principal root of $[1,0,\ldots , 0, \left\lceil \frac{L(L+1)}{4} \right\rceil +1].$ This is saying $\lambda _{L}=B_{L},$ for all $L.$
\end{itemize}
\end{frame}
\begin{frame}{Denseness of Incomplete Roots}
	\begin{theorem}[SMALL 2020]
	For any $L\in \Z ^{+}$, let $R_{L}$ be the set of roots of all incomplete PLRS of length $L$. Then, for any $\varepsilon >0,$ there exists an $M$ such that for all $L>M$, for any $\varepsilon $-ball $B_{\varepsilon }\subset [ 1,2 ]$, $B_{\varepsilon }\cap R_{L}\neq \varnothing.$
\end{theorem}
\pause
\bigskip
\begin{corollary}
	The set $R=\bigcup _{L=1}^{\infty }R_{L}$ of all principal roots of incomplete sequences is dense in $[1,2]$.
\end{corollary} 

\end{frame}
%\begin{frame}{Arbitrarily Small Incomplete Roots}

%Asymptotic work on $\lambda _{L}$ gives us useful information for the bound $B_{L}$.	
%\pause
%\medskip
%\begin{theorem}[SMALL 2020]
%For $L \in \Z _{+}$, let $\lambda _{L}$ be the sole positive root of \[
	%p_{L}(x)=x^{L}-x^{L-1}-\left\lceil \frac{L(L+1)}{4} \right\rceil 
%.\] 
%Then, for any $L,\; \lambda _{L}>\lambda _{L+1}$. Additionally, $\lim_{L \rightarrow \infty }\lambda _{L}=1.$\end{theorem}
%\pause
%Both of these results can be computed algebraically. 
%\pause
%This shows $\lim_{L \rightarrow \infty }B_{L}=1,$ so there are incomplete sequences that grow arbitrarily slowly. 

%If conjecture holds, $B _{L}\approx \left( L/2 \right) ^{2/L}$.

%\end{frame}
\begin{comment}

\begin{frame}{Proving the Conjecture}
We first show any sequence $[c_1,\ldots , c_{L}]$ where $\sum_{}^{}c_{i} $ is sufficiently large must have root greater than $\lambda _{L}.$
\pause	

\underline{Case 1}: $\sum_{k=1}^{L}c_k \geq 2+\left\lceil \frac{L(L+1) }{4} \right\rceil$

We combine the following two invariant arguments:

\pause
\begin{itemize}
\item The principal root of  $[c_1,\ldots, c_L]$ is strictly greater than that of $[c_1,\ldots,c_k-1,\ldots,c_L+1],$ for any $k$.
\pause
\item The principal root of $[1,0,\ldots,0,S]$ is strictly greater than that of $[1,0,\ldots,0,S-1]$.
\end{itemize}
\pause
Combining these two, any sequence with large sum can be "reduced" to $[1,0,\ldots,0,\left\lceil \frac{L(L+1) }{4} \right\rceil+1]$. 
\end{frame}
\begin{frame}{Inducting for the General Case}

	\begin{conjecture}
		If $[c_1,\ldots , c_{L}]$ is an incomplete sequence with $\sum_{i=1}^{L}c_{i}\leq \left\lceil \frac{L(L+1)}{4} \right\rceil +2,$ then its principal root is at least $\lambda _{L}$.
	\end{conjecture}
	\pause
\bigskip
\underline{Base Case}:	
For $L=2,$ we see $\; \left\lceil L(L+1)/4 \right\rceil +1=3$, and so we consider $[c_1,c_2]$ with $c_1+c_2\leq 4$. The only incomplete sequences here are $[2,1],[2,2],[1,3],[3,1]$, with roots $2.414,$ $ 2.731,$ $ 2.303,$ $ 3.303$. The smallest corresponds to $[1,3]=[1, \left\lceil (2\cdot 3)/4 \right\rceil +1],$ and so the Lemma holds.
\end{frame}
\begin{frame}{Inducting for the General Case}
	We use strong induction. Suppose the lemma holds for all lengths up to $L$, and let $[c_1,\ldots , c_{L},c_{L+1}]$ be an incomplete sequence with $\sum_{i=1}^{L+1}c_{i}\leq \left\lceil \frac{\left( L+1 \right) \left( L+2 \right) }{4} \right\rceil +2$.
	\pause
	\begin{itemize}
	\item
		We can show analytically that the root of $[c_1,\ldots , c_{L},c_{L+1}]$ is greater than that of $[c_1,\ldots , c_{L}]$. Thus if $[c_1,\ldots , c_{L}]$ is incomplete, its root exceeds $\lambda _{L}$ by induction hypothesis, and so the root of $[c_1,\ldots , c_{L},c_{L+1}]$ exceeds $\lambda _{L+1}.$
		\pause
	\item
		If $\sum_{i=1}^{L}c_{i}>\left\lceil L(L+1)/4 \right\rceil +2$, a similar argument shows the root of $[c_1,\ldots , c_{L},c_{L+1}]$ exceeds $\lambda _{L+1}.$ 	
		\pause
	\end{itemize}
	Thus we are reduced to the case where $[c_1,\ldots , c_{L}]$ is complete and has $\sum_{i=1}^{L}c_{i}\leq \left\lceil L(L+1)/4 \right\rceil +2$.
\end{frame}
\begin{frame}{Remaining Case}
	We we have reduced this to the case where $[c_1,\ldots , c_{L}]$ is complete and has $\sum_{i=1}^{L}c_{i}\leq \left\lceil L(L+1)/4 \right\rceil +2$, yet $[c_1,\ldots ,c_{L}, c_{L+1}]$ is incomplete. \pause As $[c_1,\ldots , c_{k}]$ has root below $\lambda _{k}$ for all $k$, we at least have that for any $1\leq k\leq L+1$,
\begin{center}\resizebox{0.4\textwidth}{!}{$
	\sum_{i=2}^{k}c_{i}\leq \left\lceil \frac{k(k+1)}{4} \right\rceil+1.
$}\end{center}
\pause
If $[c_1,\ldots , c_{L},c_{L+1}]$ is incomplete, then by previous result, $[c_1,\ldots , c_{L}+c_{L+1}]$ is incomplete too. Thus root of $[c_1,\ldots , c_{L}+c_{L+1}]$ exceeds $\lambda _{L}$, yet root of $[c_1,\ldots , c_{L}, c_{L+1}]$ is below $\lambda _{L+1}$, from which we get
\[
	\sum_{i=2}^{L}c_{i}\left( \lambda _{L+1}^{L+1-i}-\lambda _{L}^{L-i} \right) <\frac{L+2}{2}
.\] 
\end{frame}
\begin{frame}{Remaining Case}
	Using the bound
	\[
	\sum_{i=2}^{L}c_{i}\left( \lambda _{L+1}^{L+1-i}-\lambda _{L}^{L-i} \right) <\frac{L+2}{2}
,\] 
we see through asymptotic work that this forces the first $32.5 \%$ of the $c_{i}$ to be 0 (excluding $c_1$).
\pause

All experimental evidence for values of $L$ up to 30 suggest that under these conditions,  $[c_1,\ldots , c_{L},c_{L+1}]$ is only incomplete for huge values of $c_{L+1}$: much too big for the bounds on $\sum_{}^{}c_{i} $ to hold.
\pause
\[
	[1,\underbrace{0,\ldots , 0}_{19},116] \; \;  \pause \; \; [1,\underbrace{0,\ldots , 0}_{9},32,\underbrace{0,\ldots , 0}_{9},2932]
.\] 

\end{frame}
\end{comment}


\begin{comment}
\begin{frame}{Denseness of Complete Roots}
\begin{conjecture}[SMALL 2020]
	Let $C$ be the set of roots of complete PLRS. Then, $C$ is dense in the interval $(1,2)$.
\end{conjecture}
\pause
\begin{itemize}
\item
Although we have not been able to prove this rigorously, it seems that a similar argument as before is possible, only considering a different set of sequences, namely those of the form 
\end{itemize}
\[
	[1,0,\ldots , 0,\underbrace{1,\ldots , 1}_{m},N]
.\] 


\end{frame}
\end{comment}
\begin{comment}
\section{Conclusion}
\begin{frame}{Conclusion}
Here, we have developed:
\begin{itemize}
\item
	A much more computationally efficient way to check completeness for most sequences. Bounding root size is $\mathcal{O}\left( L^2 \right) $ as it amounts to evaluating polynomial, checking Brown's Criterion is a $\mathcal{O}\left( 2^{L} \right) $ problem.
\pause
\item
A narrowing-down to the precise interval where complete and incomplete sequences interact.
\pause
\item Proof that complete and incomplete sequences are evenly spread out throughout that interval.
\end{itemize}
\pause 
\textbf{Future Work}: Proving the remaining conjectures in the presentation.
\end{frame}


\section{Future Directions}
\begin{frame}{
%\begin{tikzpicture}[scale=0.5]\tikzling[signpost=\scalebox{0.6}{11},\randomhead]\end{tikzpicture}
    Future Directions}
    \begin{itemize}
    \item Extend analysis of the bound of $N$ in $[\underbrace{1,\ldots,1}_g,\underbrace{0,\ldots,0}_k,N]$, which involves the  $g$-bonacci numbers, defined by $[\underbrace{1,\dots,1}_g]$.
    \pause
    \item Find the bound $N$ for arbitrary coefficients $c_2, \dots, c_{L-1}$ in $[1, c_2, \dots, c_{L-1}, N]$.
    \pause
    \item Prove the conjectures made in this presentation.
\end{itemize}
\end{frame}
\end{comment}
\appendix

\begin{frame}{Bibliography}
\begin{thebibliography}{BBGILMT}

	% Generalized Binet's formula
	\bibitem[BBGILMT]{BBGILMT} Olivia Beckwith, Amanda Bower, Louis Gaudet, Rachel Insoft, Shiyu Li, Steven J.\ Miller, and Philip Tosteson. The Average Gap Distribution for Generalized Zeckendorf Decompositions, Dec 2012.
	
	% Brown's criterion
	\bibitem[Br]{Br} J.\ L.\ Brown. Note on complete sequences of integers. \emph{The  American  Mathematical  Monthly}, 68(6):557, 1961.
	
	\bibitem[Fr]{Fr} Aviezri S.\ Fraenkel. The use and usefulness of numeration systems. \emph{Information and Computation}, 81(1):46-61, 1989.
	
	\bibitem[MMMS]{MMMS} Thomas C.\ Martinez, Steven J.\ Miller, Clay Mizgerd, and Chenyang Sun. Generalizing Zeckendorf's Theorem to Homogeneous Linear Recurrences, 2020
	
	
\end{thebibliography}
\end{frame}

\begin{frame}{Acknowledgements}
    \begin{itemize}
        \item This research was conducted as part of the 2020 SMALL REU program at Williams College. This work was supported by NSF Grants DMS1947438 and DMS1561945, Williams College, Yale University, and the University of Rochester.
        \pause
        \item Thank you. Any questions?
    \end{itemize}
\end{frame}
\begin{comment}
\begin{frame}{Legal Decompositions vs. Completeness}
    \begin{itemize}
        %\item By Zeckendorf's Theorem, any positive integer can be written uniquely as a sum of non-adjacent Fibonacci numbers. Thus, they are also complete.
        \item Previous work on PLRS relates to \emph{legal decompositions}, which are another way to write integers as sums of sequence terms.
        \item Given any PLRS, there is a legal decomposition of every positive integer. Does this mean that all PLRS are complete?
        %All PLRS have a type of unique decomposition for all positive integers, called legal decompositions. Does this mean all PLRS are complete?
        %\pause
        \item No. For legal decompositions, sequence terms can be used more than once. This is not allowed for completeness decompositions.
        %elements of the sequence can be used multiple times.
    \end{itemize}
    
    %\pause
    
    \begin{example}
        The PLRS $[1,3]$ has terms $\{1, 2, 5, 11, \ldots\}$. The unique \emph{legal} decomposition for $9$ is $5 + 2(2)$, where the term $2$ is used twice. However, no \emph{complete} decomposition for $9$ exists.
    \end{example}
\end{frame}

\begin{frame}{Proof of Brown's Criterion}
    \begin{thm}[Brown]
	If $a_n$ is a nondecreasing sequence, then $a_n$ is complete if and only if $a_1 = 1$ and for all $n > 1$,
	\begin{equation}
	a_{n+1} \leq 1+ \sum_{i = 1}^{n} a_i. \nonumber
	\end{equation}
\end{thm}


	\textit{Proof.} Let $\{a_n\}_{n = 1}^{\infty}$ be a sequence of positive integers, not necessarily distinct, such that $a_1 = 1$ and 
	\begin{equation}
	    a_{n + 1} \leq 1 + \sum_{i = 1}^{n}{a_i} \nonumber
	\end{equation}
	for $n \in \{ 1, 2, \ldots\}$. Then for $0<n<1+ \sum_{i = 1}^{k}{a_i}$ there exists $\{b_i\}_{i = 1}^{k}$, $b_i \in \{0, 1\}$ such that $n = \sum_{i = 1}^{k}{b_i a_i}$. 
\end{frame}

\begin{frame}{Proof of Brown's Criterion}
	
	We proceed by induction on $k$. The claim clearly holds for $k = 1$, so assume that it holds for some $k = N$. Hence, we must show that $0 < n < 1 +   \sum_{i = 1}^{N + 1}{a_i}$ implies the existence of $\{\gamma_i\}^{N + 1}_{i =1}$, $\gamma_i \in \{0, 1\}$ such that $n = \sum_{i = 1}^{N + 1}{\gamma_i a_i}$. Due to the inductive hypothesis, we only consider values satisfying
	\begin{equation}
	    1 + \sum_{i = 1}^{N}{a_i} \leq n < 1 + \sum_{i = 1}^{N + 1}{a_i}. \nonumber
	\end{equation}
	Note that
	\begin{equation}
	    n-a_{N + 1} \geq 1 + \sum_{i = 1}^{N}{a_i - a_{N + 1}} \geq 0 \nonumber
	\end{equation}
    by assumption. Now, if $n - a_{N + 1} = 0$, the conclusion follows. 
    
\end{frame}
\begin{frame}{Proof of Brown's Criterion}
    
    Otherwise,
    \begin{equation}
        0 < n - a_{N + 1} < 1 + \sum_{i = 1}^{N}{a_i} \nonumber
    \end{equation}
    implies the existence of $\{b_i\}_{i = 1}^{N}$ such that $n - a_{N + 1} = \sum_{i = 1}^{N}{b_i a_i}$. Then the result is immediate on transposing $a_{N + 1}$ and identifying $\gamma_i = b_i$ for $i \in \{1, \ldots, N\}$ and $\gamma_{N + 1} = 1$. This completes the sufficiency part of the proof.
    
\end{frame}
\begin{frame}{Proof of Brown's Criterion}
    
    For the necessity, assume that there exists $n_0 \geq 1$ such that $a_{n_0 + 1} \geq 1 + \sum_{i = 1}^{n_0} a_i$. Then, however, \begin{equation}
        a_{n_0 + 1} > a_{n_0 + 1} - 1 > \sum_{i = 1}^{n_0} a_i, \nonumber
    \end{equation}
    which implies that the positive integer $a_{n_0 + 1} - 1$ cannot be represented in the form $\sum_{i = 1}^{k}{b_i a_i}$. This leads to a contradiction and completes the proof.
\end{frame}
\end{comment}

\section{Appendix}

\begin{frame}{Proof of Denseness Theorem}
We use that the $\lambda _{L}$ roots are decreasing, and $\lim_{L \rightarrow \infty }\lambda _{L}=1$.	
\pause
\begin{proof}
Consider the following incomplete sequences:
\begin{center}\resizebox{0.9\textwidth}{!}{$
	[1,0,\ldots , 0,\left\lceil \frac{L(L+1)}{2} \right\rceil +1],\; 	[1,0,\ldots , 0,\left\lceil \frac{L(L+1)}{2} \right\rceil +2],\; \ldots , 	[1,0,\ldots , 0,2^{L} ]
$}\end{center}
\pause
\begin{itemize}
\item
The root of the first sequence approaches 1. 
\item Roots of consecutive sequence increase at a decreasing rate. 
\item Root of the last sequence exceeds 2. 
\item Thus for $\lambda _{L}<1+\varepsilon $, roots are going up by at most $\varepsilon $. 
\end{itemize}
\end{proof}




\end{frame}

\end{document}