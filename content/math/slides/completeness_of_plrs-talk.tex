% based on a file Copyright 2007 by Till Tantau
% This file may be distributed and/or modified
% 1. under the LaTeX Project Public License and/or
% 2. under the GNU Public License.
\documentclass{beamer}
\setbeamertemplate{footline}[frame number]
\usetheme{Darmstadt}
\usecolortheme{rose}
\xdefinecolor{pur}{rgb}{.25,.2,.80}
\usepackage{amssymb,amsmath,amsthm}
\newcommand{\purco}[1]{{\color{pur}{#1}}}
\newcommand{\purline}{${\color{pur}\underline{\hspace{4.3in}}}$}
\newtheorem{defn}{Definition}
\newtheorem*{thm}{Theorem}
\newtheorem*{coro}{Corollary}
\newtheorem*{rmk}{Remark}
\newtheorem*{question}{Question}
\newtheorem*{conjecture}{Conjecture}

\setbeamertemplate{navigation symbols}{}
\usepackage{mathrsfs}

% Standard packages

\usepackage[english]{babel}
\usepackage[latin1]{inputenc}
\usepackage{times}
\usepackage[T1]{fontenc}
\usepackage{hyperref}
\usepackage{comment}

%\newcommand{\db}{\overline{d}}
\newcommand{\burl}[1]{\textcolor{blue}{\url{#1}}}
\newcommand{\emaillink}[1]{\textcolor{blue}{\href{mailto:#1}{#1}}}

\newcommand{\blue}[1]{ {\color[rgb]{0,0,.85} #1 } }
\newcommand{\red}[1]{ {\color[rgb]{.55,0,.15} #1 } }
\newcommand{\marroon}[1]{ {\color[rgb]{.55,.0,.5} #1 } }
\newcommand{\green}[1]{ {\color[rgb]{0,.85,0} #1 } }

%%make section title slides show up (added by Roger)

\AtBeginSection[]{
  \begin{frame}
  \vfill
  \centering
  \begin{beamercolorbox}[sep=8pt,center,shadow=true,rounded=true]{title}
    \usebeamerfont{title}\insertsectionhead\par%
  \end{beamercolorbox}
  \vfill
  \end{frame}
}

\newcommand{\mcs}{\mathcal{S}}
\newcommand{\mcd}{\mathcal{D}}
\newcommand{\mcdc}{\mathcal{D}^{\rm c}}
\newcommand{\mcsc}{\mathcal{S}^{\rm c}}
\newcommand{\mel}{\mathcal{M}}
\newcommand{\ncr}[2]{{#1 \choose #2}}
\newcommand{\un}{\text{U}}
\newcommand{\sy}{\text{USp}}
\newcommand{\usp}{\text{USp}}
\newcommand{\soe}{\text{SO(even)}}
\newcommand{\soo}{\text{SO(odd)}}
\newcommand{\so}{\text{O}}
\newcommand{\sym}{\text{sym}}

\newcommand{\fon}{\frac{1}{N}}
\newcommand{\fonk}{ \frac{1}{N^{k+1}}}
\newcommand{\glik}{\gl_i^k(A)}
\newcommand{\mtk}{M_{2k}(N)}
\newcommand{\bs}[2]{b_{|i_{#1}-i_{#2}|}}
\newcommand{\Inv}[1]{\frac{1}{#1}}


\renewcommand{\i}{{\mathrm{i}}} % sqrt -1

\newcommand{\twocase}[5]{#1 \begin{cases} #2 & \text{#3}\\ #4
&\text{#5} \end{cases}   }

\newcommand{\threecase}[7]{#1 \begin{cases} #2 & \text{{\rm #3}}\\ #4
&\text{{\rm #5}}\\ #6 & \text{{\rm #7}} \end{cases}   }


\newcommand{\lp}{\left(}
\newcommand{\rp}{\right)}


\renewcommand{\i}{{\mathrm{i}}} % sqrt -1

\newcommand\be{\begin{equation}}
\newcommand\ee{\end{equation}}
\newcommand\bea{\begin{eqnarray}}
\newcommand\eea{\end{eqnarray}}
\newcommand\bi{\begin{itemize}}
\newcommand\ei{\end{itemize}}
\newcommand\ben{\begin{enumerate}}
\newcommand\een{\end{enumerate}}


\newcommand{\notdiv}{\nmid}

\newcommand{\pr}{\mathbb{P}}
\newcommand{\R}{\mathbb{R}}
\newcommand{\C}{\mathbb{C}}
\newcommand{\Z}{\mathbb{Z}}
\newcommand{\Q}{\mathbb{Q}}
\newcommand{\N}{\mathbb{N}}

\DeclareMathOperator{\nsmod}{mod}
\newcommand{\nospacemod}[1]{\nsmod#1}

\newenvironment{variableblock}[3]{%
  \setbeamercolor{block body}{#2}
  \setbeamercolor{block title}{#3}
  \begin{block}{#1}}{\end{block}}

% Author, Title, etc.

\Large

\title[PLRS]
{
  Completeness of Positive Linear Recurrence Sequences
}

\author[Steven J Miller] %%
{\textcolor{black}{El\.zbieta Bo\l dyriew (eboldyriew@colgate.edu) \\ John Haviland (havijw@umich.edu) \\ Phuc Lam (plam6@u.rochester.edu) \\ John Lentfer (jlentfer@hmc.edu) \\ Fernando\ Trejos\ Su\'arez (fernando.trejos@yale.edu) \\[0.1in] Joint work with Steven J. Miller}\vspace{-0.25in}}

\date[2020]
{\small{\textcolor{blue}{The Nineteenth International Conference on Fibonacci Numbers and Their Applications \\ 07/21/2020}}
	%\alert{College Name, Date, Year}
}


% The main document

\begin{document}

\begin{frame}
  \titlepage
\end{frame}

\large

%%%%%%%%%%%%%%%%%%%%%%% INTRO %%%%%%%%%%%%%%%%%%%%%%%%%%%%%%%%%%%%%%%%%%%

\section{Introduction}

\begin{frame}{Positive Linear Recurrence Sequences}
    \begin{definition}
        A sequence $\{H_i\}_{i \ge 1}$ of positive integers is a {\bf Positive Linear Recurrence Sequence (PLRS)} if the following properties hold: 
        \begin{itemize}
            \pause
            \item (Recurrence relation) There are non-negative integers $L, c_1, \dots, c_L$ such that \vspace{-0.1in}
                \[ 
                    H_{n+1} = c_1H_n + \dots + c_LH_{n+1-L}
                \]
            with $L, c_1, c_L$ positive.
            \pause
            \item (Initial conditions) $H_1 = 1$, and for $1 \le n < L$, \vspace{-0.1in}
                \[
                    H_{n+1} = c_1H_n + \cdots + c_nH_1 + 1
                \]
        \end{itemize}
    \end{definition}
\end{frame}

\begin{frame}{Positive Linear Recurrence Sequences}
    \begin{itemize}
        \item We write $[c_1, \ldots ,c_L]$ for $H_{n+1} = c_1 H_n + \cdots + c_L H_{n-L+1}$.\\ \
        \pause
        %\item The definition requires that $c_1, \ldots ,c_L$ must be all non-negative, with the first and last entries positive. 
        %\item The initial conditions are specified to give unique decompositions (by a generalized Zeckendorf Theorem)
        \item For example, for the Fibonacci numbers, we write $[1,1]$. This definition gives initial conditions $F_1=1,\; F_2=2$.\\ \
        \pause
        \item Despite satisfying positive linear recurrences, the Lucas and Pell numbers are not PLRS, since their initial conditions do not meet the definition.
    \end{itemize}
\end{frame}


%\section{Completeness of PLRS}
\begin{frame}{Introduction to Completeness}
    \begin{definition}
        A sequence $\{H_i\}_{i \ge 1}$ is called \textbf{complete} if every positive integer is a sum of its terms, using each term at most once.
    \end{definition}
    \medskip
    \begin{itemize}
    \pause
     \item The sequence with the recurrence $[1, 3]$ is \emph{not} complete. Its terms are $\{1,2,5,11,\dots\}$; you cannot get $4$ or $9$ as the sequence grows too quickly.\medskip
    \pause
    \item The Fibonacci sequence $F_{n + 1} = F_n + F_{n - 1}$, with initial conditions $F_1 = 1$, $F_2 = 2$, is complete (follows from Zeckendorf's Theorem).
    \end{itemize}
    
    
\end{frame}

\begin{frame}{The Doubling Sequence}
    The PLRS $[2]$, which has the recurrence $H_{n+1} = 2H_n$, has terms $H_n = 2^{n - 1}$ and is complete because every integer has a binary representation.
    \pause
    \bigskip
    \begin{theorem}[Brown]
    The complete sequence with maximal terms is $H_n= 2^{n-1}$.
    \end{theorem}
        %\begin{theorem}
        %The sequence $a_n = 2^{n - 1}$ is the only complete sequence for which the representation of every positive integer is unique.
        %\end{theorem}
    \pause
    \bigskip
    Any PLRS of the form $[1, \ldots, 1, 2]$ has the same terms as $[2]$, i.e., $H_n = 2^{n - 1}$.
    %For any $m$, the sequence defined by $[\underbrace{1,\ldots,1}_{m},2]$ is identical to the sequence defined by $[2]$.
    %this sequence, namely $\forall n, H_n=2^{n-1}$.
\end{frame}
%\begin{frame}{Legal Decompositions vs. Completeness}
%    \begin{itemize}
        %\item By Zeckendorf's Theorem, any positive integer can be written uniquely as a sum of non-adjacent Fibonacci numbers. Thus, they are also complete.
%        \item Previous work on PLRS relates to \emph{legal decompositions}, which are another way to write integers as sums of sequence terms.
%        \item Given any PLRS, there is a legal decomposition of every positive integer. Does this mean that all PLRS are complete?
        %All PLRS have a type of unique decomposition for all positive integers, called legal decompositions. Does this mean all PLRS are complete?
%        \pause
%        \item No. For legal decompositions, sequence terms can be used more than once. This is not allowed for completeness decompositions.
        %elements of the sequence can be used multiple times.
%    \end{itemize}
    
%    \pause
    
%    \begin{example}
%        The PLRS $[1,3]$ has terms $1, 2, 5, 11, \ldots$. The unique \emph{legal} decomposition for $9$ is $5 + 2(2)$, where the term $2$ is used twice. However, no \emph{complete} decomposition for $9$ exists.
%    \end{example}
%\end{frame}

\begin{frame}{Brown's Criterion}
    \begin{theorem}[Brown]
    A nondecreasing sequence $\{H_i\}_{i \ge 1}$ is complete if and only if $H_1 = 1$ and for every $n \ge 1$,\vspace{-0.1in}
        \[
            H_{n + 1} \leq 1 + \sum_{i = 1}^{n} H_i.
        \]
    \end{theorem}
    \pause
    \bigskip
    Can we bound where a sequence must fail Brown's Criterion?
    \pause We think so!
    \bigskip
    \begin{conjecture}[SMALL 2020]
        If a PLRS $H_{n + 1} = c_1 H_n + \cdots + c_L H_{n + 1 - L}$ incomplete, then it fails Brown's criterion before the $2L$-th term.
    \end{conjecture}
\end{frame}

\section{Families of Sequences}

\begin{frame}{Analyzing Families of Sequences}
    \begin{theorem}[SMALL 2020]
        \begin{enumerate}
        \item     $[1, \underbrace{0, \ldots, 0}_k, N]$, is complete if and only if \vspace{-0.2in}
            \[
            N \leq \left\lfloor \frac{(k + 2)(k + 3)}{4} + \frac{1}{2} \right\rfloor.
            \]
        \item $[1, 1, \underbrace{0, \ldots, 0,}_k N]$, is complete if and only if
        \vspace{-0.2in}
            \[
            N\leq \left\lfloor\frac{F_{k+6}-(k+5)}{4}\right\rfloor,
            \]
            where $F_k$ is the $k$th Fibonacci number.
        \end{enumerate}
    \end{theorem}
     
\end{frame} 

\begin{frame}{Proof Sketch}
    \begin{theorem}[SMALL 2020]
        \begin{enumerate}
        \item $[1, 0, \ldots, 0, N]$, with $k$ zeros, is complete if and only if
            $N \leq \left\lfloor \frac{(k + 2)(k + 3)}{4} + \frac{1}{2} \right\rfloor$.
        \end{enumerate}
    \end{theorem}
    \emph{Partial Proof.}
        We sketch that if $N_{\max} = \left\lfloor \frac{(k + 2)(k + 3)}{4} + \frac{1}{2} \right\rfloor$, then the sequence is complete. It is similar for $N < N_{\max}$.\\
        \pause
        With the recurrence relation and Brown's Criterion,
        \begin{align*}
            H_{n+1} &=H_n + N_{\max}H_{n-k-1}\\
        &\leq H_n +(N_{\max}-1)H_{n-k-1} + H_{n-k-2} + \dots + H_1 +1
        \intertext{By induction, $(N_{\max}-1)H_{n-k-1}\leq H_{n-1} +\dots+H_{n-k-1}$, so}
         &\leq H_n + \dots + H_1 +1.
        \end{align*}
\end{frame}

\begin{frame}{Example for $L=6$}
    By the previous theorem, $[1,0,0,0,0,N]$ is complete for $N \leq 11$.\\
    \bigskip
    \pause
    
    \setbeamercolor{block}{bg=yellow, fg=white}
    \begin{variableblock}{Question}{bg=white,fg=black}{bg=purple,fg=white}
    Does there exist a complete PLRS of length $L=6$ with $N > 11$?
    \end{variableblock}
        %\item Is this the highest $N$ we can get for any PLRS of length $L=6$?
        %\item By the second part of the theorem, $[1,1,0,0,0,N]$ is complete for $N \leq 11$ by switching the first zero to a one!
\end{frame}

\begin{frame}{Example for $L=6$}
Here are the maximal last terms for preserving completeness for several other sequences of length $L=6$:
\pause
\begin{itemize}
    \item $[1,0,0,0,0,N]$ is complete for $N \leq 11$.
    \item $[1,1,0,0,0,N]$ is complete for $N \leq 11$.
    \item $[1,0,1,0,0,N]$ is complete for $N \leq 12$.
    \item $[1,0,0,1,0,N]$ is complete for $N \leq 11$.
    \item $[1,0,0,0,1,N]$ is complete for $N \leq 10$.
\end{itemize}
\pause
Why is $[1,0,1,0,0,12]$ complete, but $[1,0,0,0,0,12]$ is not complete? 
    
\end{frame}

\begin{frame}{Example for $L=6$}
Why is $[1,0,1,0,0,12]$ complete, but $[1,0,0,0,0,12]$ is not complete? \\
\begin{itemize}
\pause
\item $[1,0,0,0,0,12]$ has terms $\{1,2,3,4,5,6,18,\textcolor{red}{42},\dots \}$\\
         and so computing the sums $\sum_{i=1}^{n}H_i +1$ we see $\{2,4,7,11,16,22,\textcolor{red}{40},\dots \}$\\    
\pause
\item $[1,0,1,0,0,12]$ has terms $\{1,2,3,5,8,12,29,\textcolor{blue}{61},\dots \}$\\
         and so computing the sums $\sum_{i=1}^{n}H_i +1$ we see $\{2,4,7,12,20,32,\textcolor{blue}{61},\dots \}$\\
\pause
\item $[1,1,1,0,0,12]$ has terms $\{1, 2, 4, 8, 15, 28,\textcolor{red}{63},\dots \}$\\
         and so computing the sums $\sum_{i=1}^{n}H_i +1$ we see $\{2,4,8,16,31,\textcolor{red}{59},\dots \}$\\
\pause
% Talk about jump in sequence when last coefficient comes into play
%\item In order to maximize $N$, a complete sequence cannot grow too fast; however, by increasing intermediate terms, it is possible to give the partial sums a boost which will allow for a larger final coefficient.
\end{itemize}
\end{frame}
\begin{frame}{Sequences of Initial Ones}
\begin{theorem}[SMALL 2020]
If a sequence $[\underbrace{1,\ldots,1}_m,\underbrace{0,\ldots,0}_k,N]$ is complete with $m\geq 3,$ then
%\setlength{\abovedisplayskip}{0pt}
%\setlength{\belowdisplayskip}{0pt}
% The following line would not compile for me
%\[\resizebox{1\textwidth}{!}{N \leq \frac{1}{2} \left( 1 + \sum\limits_{i = 1}^{k + 1}F_i^{(m)} + \sum\limits_{i = 1}^{k + 1 - m}F^{(m)}_{i} + \cdots + \sum\limits_{i = 1}^{(k + 1)\nospacemod{m}}F^{(m)}_{i} \right)}\]
\[{N \leq \frac{1}{2} \left( 1 + \sum\limits_{i = 1}^{k + 1}F_i^{(m)} + \sum\limits_{i = 1}^{k + 1 - m}F^{(m)}_{i} + \cdots + \sum\limits_{i = 1}^{(k + 1)\nospacemod{m}}F^{(m)}_{i} \right)}\]
where $F_{i}^{(m)}$ is the $m$-bonacci sequence, $[\underbrace{1,\ldots,1}_m]$.
\end{theorem}
\end{frame}
\begin{frame}{Theorem on Adding Ones}
    \begin{theorem}[SMALL 2020]
\begin{itemize}
\item
For $L \geq 6$, consider the sequence $\{ H_n\}$ given by $[1,0,\dots,0,1,0,\ldots,0,M]$. Then, if  $M$ is maximal such that $\{H_n\}$ is complete, and $N$ is maximal such that 
$[1,0,\dots,0,N]$ is complete, we have $M\geq N$.
\item
For a fixed length $L$, the sequence $[1,\underbrace{0,\ldots, 0}_{k},\underbrace{1,\ldots,1}_{m},N]$ with $m$ ones has a lower bound on $N$ than the sequence $[1,\underbrace{0,\ldots, 0}_{k-1},\underbrace{1,\ldots,1}_{m+1},N]$.

In particular, if $m<\frac{L}{2}$, the bound is precisely
\begin{center}\resizebox{0.9\textwidth}{!}{$\displaystyle N \leq \left\lfloor \frac{\left( L-m \right) \left( L+m+1 \right) }{4}+\frac{1}{48}m(m+1)(m+2)(m+3)+\frac{1-2m}{2} \right\rfloor.$}\end{center}

\end{itemize}
\end{theorem}
\end{frame}
\section{Modifying Sequences}
\begin{frame}{Modifying Coefficients of a PLRS}
When studying a PLRS, what modifications to the recurrence coefficients preserve completeness or incompleteness?
\pause
\begin{theorem}[SMALL 2020]
\begin{itemize}
    \item If a sequence $[c_1, \ldots, c_{L-1}, c_L]$ is complete, then so is $[c_1, \ldots, c_{L-1}, d_L]$ for any $d_L \leq c_L$.\\
    \emph{Remark}. This is not true for $c_i$ in any position.\\
    \item If a sequence $[\underbrace{1, \ldots,1}_m,\underbrace{0,\ldots,0}_k, c_{L}]$ is complete and $c_{L}=2^{k+1}-1,$ $[\underbrace{1, \ldots,1}_m,\underbrace{0,\ldots,0}_k, c_{L}+j]$ is incomplete for any positive integer $j$.
\end{itemize}
\end{theorem}
\end{frame}
\begin{frame}{Modifying Lengths of a PLRS}
\begin{theorem}[SMALL 2020]
\begin{itemize}
\item If a sequence $[c_1, \ldots, c_L]$ is incomplete, then so is $[c_1, \ldots, c_{L-1}+c_L]$.
    \item If a sequence $[c_1, \ldots, c_{L}]$ is incomplete, then so is $[c_1, \ldots, c_{L}, c_{L+1}]$ for any $c_{L+1} > 0$.
\end{itemize}
\end{theorem}

\pause 

\begin{conjecture}[SMALL 2020]
If a sequence $[\underbrace{1, \ldots,1}_m,\underbrace{0,\ldots,0}_k, c_{L}]$  is complete, then so is  $[\underbrace{1, \ldots,1}_{m+j},\underbrace{0,\ldots,0}_k, c_{L}]$ for any positive integer $j.$
\end{conjecture}
\end{frame}

\section{Principal Roots}
\begin{frame}{Principal Roots}
    \begin{theorem}[Binet's Formula]
    If $r_1, \ldots, r_k$ are the distinct roots of the characteristic polynomial of a PLRS $\{H_n\}$, then there exist polynomials $q_1,\ldots, q_k$ such that $ H_n = q_1(n)r_1^n+ \cdots +q_k(n)r_k^n$.
    \end{theorem}
    \pause
    For PLRS, the characteristic polynomial has a unique positive root $r_1$ which is the largest in absolute value, called the \emph{principal root.} 
    \pause
    \begin{theorem}[SMALL 2020]
        If $H_n$ is a complete PLRS and $r_1$ is its principal root, then $r_1\leq 2$.
    \end{theorem}
        
\end{frame}

\begin{frame}{Bounding Principal Roots}
\begin{itemize}
\item If a sequence is complete, $r_1 \leq 2$.
\pause
\item There exists a second bound $1<B_L<2$ on the principal roots, so that if a sequence is incomplete, the  its principal root $r_1$ satisfies $r_1 \geq B_L$. This bound is dependent on the length of the generating sequence $[c_1, \ldots, c_L]$. We conjecture the following:
\begin{conjecture}[SMALL 2020]
For any given $L$, the incomplete sequence of length $L$ with the lowest principal root is $[1, 0, \ldots, 0, \left\lceil \frac{L(L+1)}{4}\right\rceil+1].$
\end{conjecture}
\pause
\item If this holds, then for large $L$, we would have $B_L \approx (L/2)^{2/L}$. In particular, $\lim_{L\rightarrow \infty}B_L=1$.
\end{itemize}
\end{frame}
\begin{frame}{Root-Bounding Proof Sketch}
\begin{conjecture}[SMALL 2020]
For any given $L$, the incomplete sequence of length $L$ with the lowest principal root is $[1, 0, \ldots, 0, \left\lceil \frac{L(L+1)}{4}\right\rceil+1].$
\end{conjecture}
\pause
Suppose $[c_1,\ldots,c_L]$ is an incomplete sequence.

\underline{Case 1}: $\sum_{k=1}^{L}c_k \geq 2+\left\lceil \frac{L(L+1) }{4} \right\rceil$

We combine the following two invariant arguments:
\pause
\begin{itemize}
\item The principal root of  $[c_1,\ldots, c_L]$ is strictly greater than that of $[c_1,\ldots,c_k-1,\ldots,c_L+1],$ for any $k$.
\pause
\item The principal root of $[1,0,\ldots,0,S]$ is strictly greater than that of $[1,0,\ldots,0,S-1]$.
\end{itemize}
\pause
Combining these two, any sequence with large sum can be "reduced" to $[1,0,\ldots,0,\left\lceil \frac{L(L+1) }{4} \right\rceil+1]$. 
\end{frame}
\begin{frame}{Root-Bounding Proof Sketch}
\underline{Case 2}: $\sum_{k=1}^{L}c_k \leq 1+\left\lceil \frac{L(L+1) }{4} \right\rceil$ 

It can be shown any ``counterexample'' would fulfill:

\begin{itemize}
\item $\forall 1\leq k\leq L+1$, \[\sum_{i=2}^{k}c_{i}\leq \left\lceil \frac{k(k+1)}{4} \right\rceil .\]

\item $\displaystyle\sum_{i=2}^{L}c_{i}\left( \lambda _{L+1}^{L+1-i}-\lambda _{L}^{L-i} \right) < \frac{L+2}{2}$,
where $\lambda_L$ is the root of $[1,0,\ldots, 0,\left\lceil L(L+1)/4 \right\rceil +1$.
\end{itemize}

This forces the coefficients of $[c_1,\ldots, c_L]$ to be small enough to force a contradiction; for example, an analytical argument shows the first $32.5\%$ or so must be $0$.

\end{frame}
\section{Future Directions}
\begin{frame}{Future Directions}
    \begin{itemize}
    \item Extend analysis of the bound of $N$ in $[\underbrace{1,\ldots,1}_m,0,\ldots,0,N]$, which involves the  $m$-bonacci numbers, defined by $[\underbrace{1,\dots,1}_m]$.
    \pause
    \item Find the bound $N$ for arbitrary coefficients $c_2, \dots, c_{L-1}$ in $[1, c_2, \dots, c_{L-1}, N]$.
    \pause
    \item Prove the conjectures made in this presentation.
\end{itemize}
\end{frame}


\begin{frame}{Bibliography}
\begin{thebibliography}{BBGILMT}


    \bibitem[MMMS]{MMMS} Thomas C.\ Martinez, Steven J.\ Miller, Clay Mizgerd, and Chenyang Sun. Generalizing Zeckendorf's Theorem to Homogeneous Linear Recurrences, 2020
	
	% Generalized Binet's formula
	\bibitem[BBGILMT]{BBGILMT} Olivia Beckwith, Amanda Bower, Louis Gaudet, Rachel Insoft, Shiyu Li, Steven J.\ Miller, and Philip Tosteson. The Average Gap Distribution for Generalized Zeckendorf Decompositions, Dec 2012.
	
	% Brown's criterion
	\bibitem[Br]{Br} J.\ L.\ Brown. Note on complete sequences of integers. \emph{The  American  Mathematical  Monthly}, 68(6):557, 1961.
	
\end{thebibliography}
\end{frame}

\begin{frame}{Acknowledgements}
    \begin{itemize}
        \item Thank you. Any questions?
    
        \item This research was conducted as part of the 2020 SMALL REU program at Williams College. This work was supported by NSF Grant DMS1947438, Williams, Yale, and Rochester.
    \end{itemize}
\end{frame}

\begin{frame}{Legal Decompositions vs. Completeness}
    \begin{itemize}
        %\item By Zeckendorf's Theorem, any positive integer can be written uniquely as a sum of non-adjacent Fibonacci numbers. Thus, they are also complete.
        \item Previous work on PLRS relates to \emph{legal decompositions}, which are another way to write integers as sums of sequence terms.
        \item Given any PLRS, there is a legal decomposition of every positive integer. Does this mean that all PLRS are complete?
        %All PLRS have a type of unique decomposition for all positive integers, called legal decompositions. Does this mean all PLRS are complete?
        %\pause
        \item No. For legal decompositions, sequence terms can be used more than once. This is not allowed for completeness decompositions.
        %elements of the sequence can be used multiple times.
    \end{itemize}
    
    %\pause
    
    \begin{example}
        The PLRS $[1,3]$ has terms $1, 2, 5, 11, \ldots$. The unique \emph{legal} decomposition for $9$ is $5 + 2(2)$, where the term $2$ is used twice. However, no \emph{complete} decomposition for $9$ exists.
    \end{example}
\end{frame}

\end{document}
